\chapter{Results}\label{chap:results}


Describe the experimental setup, the used datasets/parameters and the experimental results achieved

\section{Dataset}
In this implementation we used the following datasets: text8\footnote{matt mahoney} and ewik9\footnote{matt}, these are both repsectively the first 30 and 100MB of clean text from Wikipedia. We chose the first dataset because it is very small, hence giving us a fast computation time, but at the same time used in a lot of related work. \cite{intel} \cite{gpu}. We needed the second dataset to compute the analogy task (more on this in results). 
Both of these datasets contain documents of 10k senteces. Therefore we needed to split the dataset into sentences. We arbitrarly chose a length of 20. Furthermore we applied subsampling too both of these datasets. Statistics regarding the dataset can be found in the following subsection. 

\subsection{Specification of datasets}
\begin{table}[]
\begin{tabular}{|l|l|l|l|}
\hline
Dataset & Voc size & Numb of sentences (20) & Number of Sentences(length 20) with sampling \\ \hline
text8   & 250k     & 4000                   & 2000                                         \\ \hline
enwik9  & 750k     & 100000                 & 5000                                         \\ \hline
\end{tabular}
\caption{Specifications of the dataset}
\end{table}

\section{Configuration of the network}
The skip-gram model contains a lot of hyper parameters that can be tuned. 
Let's have a look at theim and how they potentially could influence the network
\begin{itemize}
\item Negative Samples: the recommended range in \cite{mikolov} is 10-20. The training accuracy should increase as the sample increases, therefore the computation time should go up. There is a tradeoff to to find. 
\item Context Window: The bigger the window the more training examples the network will have, but if the window is to big the semantic meaning of the window will be erased. The recommended setting was 2-10. 
\item Dimension of the embedding: here the choice is less obvious, in their paper mikolov et al choosed an embedding size of 300. their dataset was pretty huge, with 1 billion words. We therefore choosed a smaller one. 
\item Batch size: more on the batch size in section intel description. Too big of a batch size will lead to too much overwirting and therefor not getting the best resulsts. If we turn dhte batch size down too much we will get a higher computation time. 
\item Alpha: learnng rate. We tried out a variety of lr in the range of 0.00001 to 2 
\end{itemize}